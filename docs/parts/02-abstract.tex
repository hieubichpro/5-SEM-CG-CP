\section*{\centering РЕФЕРАТ}
\addcontentsline{toc}{section}{РЕФЕРАТ}
\setcounter{page}{2}

Расчетно-пояснительная записка \pageref{LastPage} с., \totalfigures\ рис., 4 лист., 6 ист., 1 прил.

КОМПЬЮТЕРНАЯ ГРАФИКА, АЛГОРИТМЫ УДАЛЕНИЯ НЕВИДИМЫХ ЛИНИЙ, Z-БУФЕР, ЗАКРАСКА, ОСВЕЩЕНИЕ, СЦЕНА, НАЛОЖЕНИЕ ТЕКСТУР, ПРОЦЕДУРНОЕ ТЕКСТУРИРОВАНИЕ, ПРОЕКТИВНОЕ ТЕКСТУРИРОВАНИЕ, НЕРОВНОСТЬ

Целью работы является разработка программного обеспечения для наложения текстур на трёхмерные объекты.
Для визуализации сцены использовался алгоритм с Z-буфером, а для представления объекта использовалась поверхностная модель, представленная в виде списка ребер.

В процессе работы были проанализированы различные алгоритмы, методы представления, закраски геометрических моделей на сцена. Выбраны технологии решения для поставленной задачи, а также разработаны алгоритмы для их программной реализации. Разработана программа, предназначенная для наложения текстур на трёхмерные объекты.

Проведено исследование быстродействия программы при различном количестве создания объектов на сцене. 
Также с фиксированным количеством объектов, но с учётом текстуры. 
Из результатов исследования следует, что время отрисовки сцены увеличивается как при увеличении количества объектов на сцена, так и при выборе текстуры.